Durant notre première séance de laboratoire pour le cours de physique des télécommunications, nous avons dimensionné notre antenne patch à l'aide du logiciel FEKO. Pour cela nous avons procédé par étapes, partant d'un design extrêmement simple auquel nous avons petit à petit ajouté ou modifié des éléments pour arriver à la version finale de notre antenne.
A la fin de la séance, notre antenne avait un coefficient de réflexion minimal de \SI{-27.43}{\deci\bel} à la fréquence de \SI{2.398}{\giga\hertz} là où le cahier des charges nous imposait un coefficient de réflexion de \SI{-6}{\deci\bel} à la fréquence d'utilisation de l'antenne, c'est-à-dire \SI{2.4}{\giga\hertz}.

Dans ce chapitre, nous allons détailler les différentes étapes qui nous ont amené au dimensionnement final de notre antenne.


\subsection{Antenne sur un diélectrique infini}
Pour commencer, nous avons simplement simulé un patch rectangulaire posé sur un matériau diélectrique de même permittivité électrique que le PCB utilisé en pratique pour fabriquer notre antenne. Pour ce qui est des dimensions (longueur et largeur) du patch, nous avons utilisé les formules qui nous étaient fournies. La figure \ref{fig:rayonnement_11} nous donne la directivité ainsi que le gain de l'antenne pour des valeurs de $\phi$ de \SI{0}{\degree} et \SI{90}{\degree}.
\begin{figure}[htbp]
  \centering
  \includegraphics[width=\textwidth]{rayonnement11_annotation.pdf}
  \caption{Diagramme de rayonnement du gain [généré avec PostFeko]\label{fig:rayonnement_11}}
\end{figure}
Pour les deux valeurs de $\phi$, la directivité maximale est de \SI{0}{\degree}.

Nous nous sommes aussi intéressés au coefficient de réflexion de l'antenne ainsi qu'à sa fréquence de résonance.
\begin{figure}[htbp]
  \centering
  \includegraphics[width=\textwidth]{reflection11_annotation.pdf}
  \caption{Coefficient de réflexion en fonction de la fréquence [généré avec PostFeko]\label{fig:reflection11_}}
\end{figure}
La figure \ref{fig:reflection11_} nous montre que la résonance de l'antenne se situe à \SI{2.37941}{\giga\hertz} qu'à cette fréquence le coefficient de réflexion vaut \SI{-21.8}{\deci\bel}. A \SI{2.4}{\giga\hertz} ce coefficient vaut \SI{-6}{\deci\bel}, il est intéressant de noter la valeur de la bande passante définie à \SI{-6}{\deci\bel} qui vaut ici pas loin de \SI{0.04}{\giga\hertz}. Un dernier aspect important pour ce point est le pourcentage de puissance délivrée à l'antenne à la fois à la fréquence de résonance et à \SI{2.4}{\giga\hertz}. Ce pourcentage nous est donné par la formule \ref{eqn:puissance délivrée}, ce qui nous donne une valeur de \SI{99.3}{\percent} à la fréquence de résonance et \SI{75.2}{\percent} à \SI{2.4}{\giga\hertz}.
\begin{equation}
\frac{P_L}{P_{in}} = 1-\Gamma_L^2
\label{eqn:puissance délivrée}
\end{equation}

%Note pour jojol: coeff de réflection c'est gamma majuscule, aka \Gamma_L (L pour load).
%Merci d'utiliser \SI! c'est vachement cool!
%Je me suis planté, la proportion absorbée c'est 1-(gamma)². Je m'occupe (ou me suis occupé) de corriger dans ce qui est déjà fait.
%Largeur des figures = \textwidth? Sinon je trouve qu'elles ne sont pas super lisibles
%NOTE POUR TWERKYTWERK : pour la largeur des figures je me suis 100pourcent inspiré des labos de thermo, je te laisse modifier la mise en forme à ta guise, je touche pas grand chose à ça

\subsection{Antenne sur un diélectrique fini}
L'étape suivante consiste simplement à remplacer le substrat infini par un carré de coté \SI{50}{\milli\meter}, c'est-à-dire la dimension du PCB de notre antenne. Ci-dessous, nous détaillons les changements que cette modification apporte aux différents paramètres déjà étudiés plus haut.

Premièrement, comme nous pouvons le voir sur la figure \ref{fig:reflection12_annotation}, la fréquence de résonance est passée à \SI{2.27}{\giga\hertz} où son coefficient de réflexion ne vaut plus que \SI{-8.47}{\deci\bel}. La bande de fréquences où le coefficient de réflexion vaut moins de \SI{6}{\deci\bel} vaut quand à elle \SI{0.033}{\giga\hertz}.
\begin{figure}[htbp]
\centering
\includegraphics[width=\textwidth]{reflection12_annotation.pdf}
\caption{Coefficient de réflexion en fonction de la fréquence\label{fig:reflection12_annotation}}
\end{figure}
A nouveau, il est intéressant d'étudier le diagramme de rayonnement du gain de l'antenne, donné à la figure \ref{fig:rayonnement12_annotation}, où l'on remarque que la directivité maximale est à  \SI{0}{\degree} et le gain maximal à \SI{3.64} que ce soit pour $\phi = \SI{0}{\degree} ou  \SI{90}{\degree}$.
\begin{figure}
\centering
\includegraphics[width=\textwidth]{rayonnement12_annotation}
\caption{Diagramme de rayonnement de l'antenne sur diélectrique fini}
\label{fig:rayonnement12_annotation}
\end{figure}
Pour terminer, nous avons modifié le déphasage de sorte que $\tan{\delta} = 0.01$. Le gain diminue alors à \SI{2.64} et le coefficient de réflexion vaut \SI{-6}{\deci\bel} à la résonance, qui reste inchangée. (\ref{fig:reflection122_annotation}).
\begin{figure}[htbp]
\centering
\includegraphics[width=\textwidth]{reflection122_annotation.pdf}
\caption{Coefficient de réflexion en fonction de la fréquence, avec déphasage}
\label{fig:reflection122_annotation}
\end{figure}


\subsection{Antenne avec fente}
Comme l'indique le titre, l'étape suivant a été de "creuser" une fente dans le carré de notre antenne patch. La seule consigne à suivre était que le coefficient de réflexion à la résonance devait être inférieur à \SI{-10}{\deci\bel}. Comme le montre la figure \ref{fig:reflection13_annotation}, notre choix de \SI{8}{\milli\meter} pour $y_0$ respecte cette consigne.
\begin{figure}[htbp]
\centering
\includegraphics[width=\textwidth]{reflection13_annotation.pdf}
\caption{Coefficient de réflexion en fonction de la fréquence, avec fente de \SI{8}{\milli\meter}}
\label{fig:reflection13_annotation}
\end{figure}

\subsection{Notre antenne}
Pour terminer, nous avons ajouté le microstrip dans la fente de notre patch. Mais ni le coefficient de réflexion, ni la fréquence de résonance ne répondaient au cahier des charges, nous avons donc du modifier les paramètres de notre antenne de manière à s'approcher le plus possible du résultat demandé.
\begin{figure}[htbp]
\centering
\includegraphics[width=\textwidth]{rayonnement_annotationFinal.pdf}
\caption{Diagramme de rayonnement de notre antenne}
\label{fig:rayonnement_annotationFinal}
\end{figure}
Comme on peut le voir sur la figure \ref{fig:rayonnement_annotationFinal}, notre gain maximum vaut \SI{2.25} pour une directivité de \SI{0}{\degree}. Le coefficient de réflexion est quand à lui donné à la figure \ref{fig:reflection_annotationFinal}.
\begin{figure}[htbp]
\centering
\includegraphics[width=\textwidth]{reflection_annotationFinal.pdf}
\caption{Coefficient de réflexion de notre antenne en fonction de la fréquence}
\label{fig:reflection_annotationFinal}
\end{figure}


%Bonjour, bien dormi?
%Aussi étonnant que ça puisse paraître, \includegraphics[width = x \textwidth]{file.pdf} veut dire que tu redimensionne la figure pour qu'elle soit aussi large que x fois la largeur du texte à cet endroit. A toi d'ajuster à ta guise. Mais ne t'en fais pas, je passerai quand même derrière pour râler malgré tout. U b saf mah nig